% Options for packages loaded elsewhere
\PassOptionsToPackage{unicode}{hyperref}
\PassOptionsToPackage{hyphens}{url}
%
\documentclass[
]{book}
\usepackage{amsmath,amssymb}
\usepackage{lmodern}
\usepackage{iftex}
\ifPDFTeX
  \usepackage[T1]{fontenc}
  \usepackage[utf8]{inputenc}
  \usepackage{textcomp} % provide euro and other symbols
\else % if luatex or xetex
  \usepackage{unicode-math}
  \defaultfontfeatures{Scale=MatchLowercase}
  \defaultfontfeatures[\rmfamily]{Ligatures=TeX,Scale=1}
\fi
% Use upquote if available, for straight quotes in verbatim environments
\IfFileExists{upquote.sty}{\usepackage{upquote}}{}
\IfFileExists{microtype.sty}{% use microtype if available
  \usepackage[]{microtype}
  \UseMicrotypeSet[protrusion]{basicmath} % disable protrusion for tt fonts
}{}
\makeatletter
\@ifundefined{KOMAClassName}{% if non-KOMA class
  \IfFileExists{parskip.sty}{%
    \usepackage{parskip}
  }{% else
    \setlength{\parindent}{0pt}
    \setlength{\parskip}{6pt plus 2pt minus 1pt}}
}{% if KOMA class
  \KOMAoptions{parskip=half}}
\makeatother
\usepackage{xcolor}
\IfFileExists{xurl.sty}{\usepackage{xurl}}{} % add URL line breaks if available
\IfFileExists{bookmark.sty}{\usepackage{bookmark}}{\usepackage{hyperref}}
\hypersetup{
  pdftitle={Data Dictionary for ALSPAC data: K01MH123914},
  pdfauthor={Katherine Schaumberg, Max Frank},
  hidelinks,
  pdfcreator={LaTeX via pandoc}}
\urlstyle{same} % disable monospaced font for URLs
\usepackage{color}
\usepackage{fancyvrb}
\newcommand{\VerbBar}{|}
\newcommand{\VERB}{\Verb[commandchars=\\\{\}]}
\DefineVerbatimEnvironment{Highlighting}{Verbatim}{commandchars=\\\{\}}
% Add ',fontsize=\small' for more characters per line
\usepackage{framed}
\definecolor{shadecolor}{RGB}{248,248,248}
\newenvironment{Shaded}{\begin{snugshade}}{\end{snugshade}}
\newcommand{\AlertTok}[1]{\textcolor[rgb]{0.94,0.16,0.16}{#1}}
\newcommand{\AnnotationTok}[1]{\textcolor[rgb]{0.56,0.35,0.01}{\textbf{\textit{#1}}}}
\newcommand{\AttributeTok}[1]{\textcolor[rgb]{0.77,0.63,0.00}{#1}}
\newcommand{\BaseNTok}[1]{\textcolor[rgb]{0.00,0.00,0.81}{#1}}
\newcommand{\BuiltInTok}[1]{#1}
\newcommand{\CharTok}[1]{\textcolor[rgb]{0.31,0.60,0.02}{#1}}
\newcommand{\CommentTok}[1]{\textcolor[rgb]{0.56,0.35,0.01}{\textit{#1}}}
\newcommand{\CommentVarTok}[1]{\textcolor[rgb]{0.56,0.35,0.01}{\textbf{\textit{#1}}}}
\newcommand{\ConstantTok}[1]{\textcolor[rgb]{0.00,0.00,0.00}{#1}}
\newcommand{\ControlFlowTok}[1]{\textcolor[rgb]{0.13,0.29,0.53}{\textbf{#1}}}
\newcommand{\DataTypeTok}[1]{\textcolor[rgb]{0.13,0.29,0.53}{#1}}
\newcommand{\DecValTok}[1]{\textcolor[rgb]{0.00,0.00,0.81}{#1}}
\newcommand{\DocumentationTok}[1]{\textcolor[rgb]{0.56,0.35,0.01}{\textbf{\textit{#1}}}}
\newcommand{\ErrorTok}[1]{\textcolor[rgb]{0.64,0.00,0.00}{\textbf{#1}}}
\newcommand{\ExtensionTok}[1]{#1}
\newcommand{\FloatTok}[1]{\textcolor[rgb]{0.00,0.00,0.81}{#1}}
\newcommand{\FunctionTok}[1]{\textcolor[rgb]{0.00,0.00,0.00}{#1}}
\newcommand{\ImportTok}[1]{#1}
\newcommand{\InformationTok}[1]{\textcolor[rgb]{0.56,0.35,0.01}{\textbf{\textit{#1}}}}
\newcommand{\KeywordTok}[1]{\textcolor[rgb]{0.13,0.29,0.53}{\textbf{#1}}}
\newcommand{\NormalTok}[1]{#1}
\newcommand{\OperatorTok}[1]{\textcolor[rgb]{0.81,0.36,0.00}{\textbf{#1}}}
\newcommand{\OtherTok}[1]{\textcolor[rgb]{0.56,0.35,0.01}{#1}}
\newcommand{\PreprocessorTok}[1]{\textcolor[rgb]{0.56,0.35,0.01}{\textit{#1}}}
\newcommand{\RegionMarkerTok}[1]{#1}
\newcommand{\SpecialCharTok}[1]{\textcolor[rgb]{0.00,0.00,0.00}{#1}}
\newcommand{\SpecialStringTok}[1]{\textcolor[rgb]{0.31,0.60,0.02}{#1}}
\newcommand{\StringTok}[1]{\textcolor[rgb]{0.31,0.60,0.02}{#1}}
\newcommand{\VariableTok}[1]{\textcolor[rgb]{0.00,0.00,0.00}{#1}}
\newcommand{\VerbatimStringTok}[1]{\textcolor[rgb]{0.31,0.60,0.02}{#1}}
\newcommand{\WarningTok}[1]{\textcolor[rgb]{0.56,0.35,0.01}{\textbf{\textit{#1}}}}
\usepackage{longtable,booktabs,array}
\usepackage{calc} % for calculating minipage widths
% Correct order of tables after \paragraph or \subparagraph
\usepackage{etoolbox}
\makeatletter
\patchcmd\longtable{\par}{\if@noskipsec\mbox{}\fi\par}{}{}
\makeatother
% Allow footnotes in longtable head/foot
\IfFileExists{footnotehyper.sty}{\usepackage{footnotehyper}}{\usepackage{footnote}}
\makesavenoteenv{longtable}
\usepackage{graphicx}
\makeatletter
\def\maxwidth{\ifdim\Gin@nat@width>\linewidth\linewidth\else\Gin@nat@width\fi}
\def\maxheight{\ifdim\Gin@nat@height>\textheight\textheight\else\Gin@nat@height\fi}
\makeatother
% Scale images if necessary, so that they will not overflow the page
% margins by default, and it is still possible to overwrite the defaults
% using explicit options in \includegraphics[width, height, ...]{}
\setkeys{Gin}{width=\maxwidth,height=\maxheight,keepaspectratio}
% Set default figure placement to htbp
\makeatletter
\def\fps@figure{htbp}
\makeatother
\setlength{\emergencystretch}{3em} % prevent overfull lines
\providecommand{\tightlist}{%
  \setlength{\itemsep}{0pt}\setlength{\parskip}{0pt}}
\setcounter{secnumdepth}{5}
\usepackage{booktabs}
\usepackage{amsthm}
\makeatletter
\def\thm@space@setup{%
  \thm@preskip=8pt plus 2pt minus 4pt
  \thm@postskip=\thm@preskip
}
\makeatother
\usepackage{booktabs}
\usepackage{longtable}
\usepackage{array}
\usepackage{multirow}
\usepackage{wrapfig}
\usepackage{float}
\usepackage{colortbl}
\usepackage{pdflscape}
\usepackage{tabu}
\usepackage{threeparttable}
\usepackage{threeparttablex}
\usepackage[normalem]{ulem}
\usepackage{makecell}
\usepackage{xcolor}
\ifLuaTeX
  \usepackage{selnolig}  % disable illegal ligatures
\fi
\usepackage[]{natbib}
\bibliographystyle{apalike}

\title{Data Dictionary for ALSPAC data: K01MH123914}
\author{Katherine Schaumberg, Max Frank}
\date{2022-04-21}

\begin{document}
\maketitle

{
\setcounter{tocdepth}{1}
\tableofcontents
}
\hypertarget{introduction}{%
\chapter{Introduction}\label{introduction}}

Driven exercise (DEx) is a common, debilitating symptom across eating disorders (ED). Up to 40\% of individuals with bulimia nervosa and 80\% of those with anorexia nervosa experience driven exercise. Driven exercise relates to high levels of ED symptoms and poor ED treatment outcomes, and has been purported to be an early ED symptom via retrospective reports.

A previous study examined exercise for weight loss and driven exercise at age 14 in the ALSPAC cohort, identifying whether groups at age 14 were associated with ED behaviors at ages 14 and 16 \citep{schaumberg2022}. Results found that both exercise for weight loss and driven exercise groups at age 14 were demonstrated higher levels of other ED behaviors (binge eating, fasting, purging) at age 16.

In the current line of research, we are extend a longitudinal investigation of exercise for weight loss and driven exercise across a larger developmental window (ages 14-24) in the ALSPAC Cohort.

Aims of the research include:

\begin{table}
\centering
\begin{tabular}{>{}l||>{}l|l}
\hline
Aim & Goal & Hypothesis\\
\hline
\textbf{1} & \cellcolor{gray}{\textcolor{white}{Identify and characterize the physical activity trajectories from late childhood through emerging adulthood in ALSPAC and capture associations with driven exercise and eating disorder risk.}} & High levels of activity in childhood and early adolescence will associate with both DEx and ED risk in adolescence\\
\hline
\textbf{2} & \cellcolor{gray}{\textcolor{white}{Determine the strength of DEx as a predictor of eating disorder pathology}} & DEx will present early in the course of EDs, with peak age of onset in early-to-mid adolescence, and will predict ED onset and maintenance beyond other risk factors (e.g. dieting, thin-deal internalization)\\
\hline
\textbf{3} & \cellcolor{gray}{\textcolor{white}{Demonstrate the influence of genomic risk factors on DEx}} & 'Anthropometric/Activity' and 'Compulsivity' genomic risk factors will be identified and will predict DEx\\
\hline
\end{tabular}
\end{table}

\hypertarget{the-alspac-cohort}{%
\chapter{The ALSPAC Cohort}\label{the-alspac-cohort}}

The ALSPAC Cohort \citep{boydCohortProfileChildren2013, fraserCohortProfileAvon2013} was established to understand how genetic and environmental characteristics influence health and development in parents and children. Ethical approval for this study was granted by the ALSPAC Law and Ethics Committee and Local Ethics Committees. All pregnant women living in the geographical area of Avon, United Kingdom, who were expected to deliver between April 1, 1991 and December 31, 1992, were invited to participate in the study. Children from 14,541 pregnancies were enrolled; 13,988 children were alive at 1 year. An additional 913 children were enrolled during subsequent phases of enrollment, with a total sample size alive at 1 year of 14,901. All women gave informed and written consent. Among twin pairs, one twin per pair was randomly excluded from the current study.

A fully searchable ALSPAC data dictionary is available \href{http://www.bris.ac.uk/alspac/researchers/data-access/data-dictionary/}{here}.

\hypertarget{scoring}{%
\chapter{Scoring}\label{scoring}}

Data cleaning and scoring procedures were completed primarily with the \texttt{scorekeeper} R package.

\hypertarget{eating-disorder-cognitions}{%
\chapter{Eating Disorder Cognitions}\label{eating-disorder-cognitions}}

Eating disorder cognitions were assessed at an in-depth level at age 14 years, with some additional eating disorder cognition variables assessed at other time points.

\begin{Shaded}
\begin{Highlighting}[]
\CommentTok{\#insert table here with constructs, ages, and checkmarks}
\end{Highlighting}
\end{Shaded}

\hypertarget{dieting}{%
\section{Dieting}\label{dieting}}

\hypertarget{background}{%
\subsection{Background}\label{background}}

Dieting was assessed at ages 14 and 16 years

the dieting at age 14 questionare is a set of two questions asked as a part of the ALSPAC study. Data from these questions has been used in multiple studies\ldots(insert links). The two questions asked aim to learn the extent of dieting a respondent has taken part in. The first question aims to find the frequency of dieting and the second question asks about the length of diets in the past year.

\hypertarget{scoring-1}{%
\subsection{Scoring}\label{scoring-1}}

\hypertarget{key-variables}{%
\subsection{Key Variables}\label{key-variables}}

\hypertarget{thin-ideal-internalization}{%
\section{Thin-ideal Internalization}\label{thin-ideal-internalization}}

\hypertarget{eating-disorder-behaviors}{%
\chapter{Eating Disorder Behaviors}\label{eating-disorder-behaviors}}

Eating Disorder Behaviors Were Assessed at Ages 14, 16, 18, and 24\ldots{}

\hypertarget{binge-eating}{%
\section{Binge Eating}\label{binge-eating}}

\hypertarget{background-1}{%
\subsection{Background}\label{background-1}}

Binge eating behavior was evaluated using questions derived from the Youth Risk Behavior Surveillance Scale \href{https://pubmed.ncbi.nlm.nih.gov/8981266/}{(YBSSR)} at ages 14, 16, 18, and 24. Binge Eating Beahvior in the ALSPAC data at age 14 has previously been scored and examined in several papers. At ages 14, 16, and 18, Binge eating was assessed using a two-part question. Participants were first asked about the frequency during the past year of eating a very large amount of food. Those who answered yes were asked a follow-up question that asked whether they felt out of control during these episodes, i.e., whether they could not stop eating even if they wanted to stop, along with five other questions regarding charachterization of binge eating episodes (e.g.~how often they felt guilty after a binge, how often they ate fast or faster than normal during a binge)

\href{https://pubmed.ncbi.nlm.nih.gov/26193063/}{PMID: 26193063}

\href{https://pubmed.ncbi.nlm.nih.gov/33644868/}{PMID: 33644868}

\href{https://pubmed.ncbi.nlm.nih.gov/26098685/}{PMID: 26098685}

\hypertarget{scoresheet}{%
\section{Scoresheet}\label{scoresheet}}

The binge eating scoresheet can be found {[}here{]}

To clean the data, the scoresheet:

\begin{enumerate}
\def\labelenumi{(\arabic{enumi})}
\item
  selects only the variables that are relevant for the current measure
\item
  appropriately accounts for the skip patterns in the data (recoding to `no' follow-up questions for individuals who reported no binge eating)
\item
  recodes the variables to sensible frequencies (e.g.~changing a `no' answer to be represented as `0' with increasing frequency incrementing from zero) and computes dichotomized versions of the presence vs.~absence of symtpoms for use in symtpom counts.
\item
  Creates a symptom sum score, which gives a count of the number of binge eating symptoms (0-6) that are present for each individual
\item
  Creates two additional variables based on the sum score. The first variable computes a variable that notes binge eating is present (`1') if an individual reports binge eating at least once per month with more than two additional symptoms. The second splits cases into `absent', `mild', and `severe' based on the number of symptoms reported. The tibble is then filtered to exclude cases with all missing data.
\item
  Select only a few columns that will go into the final dataset
\item
  Renames \texttt{binge\_sx\_sum\_split.14} to a simpler name: \texttt{binge\_severity.14}
\end{enumerate}

\hypertarget{key-variables-1}{%
\section{Key Variables}\label{key-variables-1}}

\texttt{binge\_severity.14}(defines the severity of binge eating at age 14 based on number of symptoms)

\texttt{binge\_present.14}(defines whether diagnostic level binge eating is present or absent)

\hypertarget{compensatory-behaviors}{%
\section{Compensatory Behaviors}\label{compensatory-behaviors}}

\hypertarget{anthropometric-variables}{%
\chapter{Anthropometric Variables}\label{anthropometric-variables}}

Anthropometric variables were collected and/or reported at several time points during development. The following time points are avaialble in this project.

\begin{Shaded}
\begin{Highlighting}[]
\CommentTok{\#insert table with anthropometric variables and timepoints assessed}
\end{Highlighting}
\end{Shaded}

\hypertarget{body-mass-index-bmi}{%
\section{Body Mass Index (BMI)}\label{body-mass-index-bmi}}

\hypertarget{dexa-scans}{%
\section{DEXA Scans}\label{dexa-scans}}

\hypertarget{physical-activity}{%
\chapter{Physical Activity}\label{physical-activity}}

\hypertarget{free-living-activity}{%
\section{Free-living activity}\label{free-living-activity}}

Free-living activity was assessed via accelerometer at ages\ldots.

\hypertarget{fitness}{%
\section{Fitness}\label{fitness}}

Fitness was assessed at xxx exercise sessions xxx\ldots.

\hypertarget{sociodemographic-variables}{%
\chapter{Sociodemographic Variables}\label{sociodemographic-variables}}

\hypertarget{parent-education}{%
\section{Parent Education}\label{parent-education}}

\hypertarget{parent-occupational-class}{%
\section{Parent Occupational Class}\label{parent-occupational-class}}

\hypertarget{child-ethnicity}{%
\section{Child Ethnicity}\label{child-ethnicity}}

\hypertarget{parent-ethnicity}{%
\section{Parent Ethnicity}\label{parent-ethnicity}}

\hypertarget{maternal-eating-disorder-assessments}{%
\chapter{Maternal Eating Disorder Assessments}\label{maternal-eating-disorder-assessments}}

Mothers reported on eating disorder cognitions and behaviors\ldots\ldots..

\hypertarget{genomic-data}{%
\chapter{Genomic Data}\label{genomic-data}}

  \bibliography{ALSPAC\_bookdown.bib,packages.bib}

\end{document}
